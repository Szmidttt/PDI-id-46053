\chapter{Podsumowanie}
Niniejsza praca miała na celu praktyczną weryfikację możliwości migracji fragmentu istniejącego kodu w języku C++ do języka Rust w kontekście dużego, rozwijanego od wielu lat projektu, jakim jest przeglądarka Mozilla Firefox. Przedmiotem badań została wybrana biblioteka testcrasher, wykorzystywana w procesie testowania mechanizmów raportowania awarii. Komponent ten jest istotnym elementem zestawu narzędzi deweloperskich wykorzystywanych w projekcie, a jednocześnie na tyle izolowanym, aby jego migracja mogła zostać przeprowadzona w sposób bezpieczny i pozwalała na ocenę praktycznych wyników.
\newline
Przeprowadzona analiza istniejących rozwiązań wskazała, że Rust od kilku lat zyskuje na popularności jako język zapewniający wysoki poziom bezpieczeństwa pamięci oraz poprawę stabilności dużych systemów. Projekty takie jak Stylo, Servo czy inicjatywy Mozilli w zakresie Oxidation potwierdzają, że Rust może latwo zastępować fragmenty kodu C++ w złozonych systemach. Jednocześnie narzędzia takie jak C2Rust, Bindgen czy Corrode wspierają proces stopniowego wdrażania języka Rust oraz utrzymania kompatybilności z istniejącymi komponentami.
\newline
Proces migracji biblioteki testcrasher obejmował kluczowe elementy jej działania, a głównym wyzwaniem była konieczność zachowania pełnej zgodności funkcjonalnej oraz zachowanie stabilnosci binarnej(ABI). Zastosowana iteracyjna, stopniowa strategia umożliwiła płynne zastępowanie funkcji C++ kodem Rust.
\newline
Język programowania Rust eliminuje wiele problemów typowych dla C++, w szczególności związanych z zarządzaniem pamięcią. To wynika z modelu własności oraz systemu typów wymuszającego jawne traktowanie błędów. Testy funkcjonalne wykazały pełną zgodność działania implementacji po migracji z wersją oryginalną, a różnice w rozmiarze biblioteki i czasie kompilacji nie wpływają na funkcjonalność systemu.
\newline
Zrealizowana praca dowodzi, że stopniowa migracja modułów C++ do Rust w dużym systemie jest możliwa i przynosi wymierne korzyści. Rust zwiększa bezpieczeństwo pamięci oraz minimalizuje ryzyko błędów trudnych do wykrycia w językach niższego poziomu. Wyniki te potwierdzają wybrany  w ostatnich latach przez Mozilla oraz inne organizacje kierunek.
W przyszłości biblioteka testcrasher zaimplementowana w Rust mogłaby zostać rozszerzona o kolejne warstwy funkcjonalne. Dalsza integracja Rust w większej liczbie modułów Firefox mogłaby zwiększyć stabilność i jakość całego oprogramowania. Wyniki przedstawione w pracy stanowią podstawę do kontynuacji badań nad wdrażaniem Rust w wieloletnich i dużych projektach programistycznych.
