\chapter{Migracja fragmentu kodu z języka programowania C++ na język Rust}
Niniejszy rozdział szczegółowo opisuje proces migracji wybranego komponentu z języka programowania C++ na język Rust. Zgodnie z założeniami projektu, poszukiwania odpowiedniego fragmentu kodu ograniczono do oprogramowania rozwijanego przez Fundację Mozilla, co w praktyce skierowało uwagę na bazę kodu przeglądarki Firefox. Główną motywacją dla podjętych działań jest dążenie do poprawy bezpieczeństwa pamięci i ogólnej stabilności aplikacji, co jest jednym ze strategicznych celów wykorzystania języka Rust w dojrzałych projektach. W dalszej części rozdziału przedstawiono kolejne etapy pracy: począwszy od kryteriów, które zadecydowały o wyborze komponentu, przez jego szczegółową analizę, aż po opis przyjętej strategii migracji, wykorzystanych narzędzi i finalnego przebiegu implementacji.
\section{Kryteria wyboru fragmetu kodu do migracji}
Podczas wyboru fragmentu kodu kierowaliśmy się kilkoma kluczowymi kryteriami:
\begin{itemize}
    \item \textbf{Wysoki potencjał poprawy bezpieczeństwa:}
    \begin {itemize}
        \item \textbf{Zarządznie pamięcią:} Kandydat do migracji powinien operować w obszarze, w którym błędy zarządzania pamięcią, typowe dla języka C++, mogą prowadzić do poważnych luk w zabezpieczeniach. Przykładem może być praca z surowymi danymi, takimi jak zrzuty pamięci.
        \item \textbf{Zastąpienie przestarzałej zależności:} Komponent opiera się na zewnętrznej bibliotece C++, uznawanej za przestarzałą lub posiadającej nowocześniejszy i bezpieczniejszy odpowiednik w języku Rust.
    \end{itemize}
        \item \textbf{Wykonalność i modularność:} Fragment kodu musiał być na tyle odizolowany, aby jego migracja nie pociągała za sobą konieczności przepisywania znacznych części przeglądarki. Biblioteka o jasno zdefiniowanym API i konkretnym zadaniu ułatwia proces zastępowania implementacji bez naruszania reszty systemu.
        \item \textbf{Zgodność ze strategicznymi celami Fundacji Mozilla:} Wybrany komponent powinien wpisywać się w długofalową strategię Mozilli polegającą na stopniowym zwiększaniu ilości kodu napisanego w Rust w celu poprawy bezpieczeństwa i wydajności przeglądarki. Zgodnie z incjatywą "Oxidation". \cite{oxidation}
    \end{itemize}
\section{Wybrany fragment kodu - biblioteka dynamiczna testcrasher}
Pierwszym etapem prac było zidentyfikowanie w kodzie źródłowym należącym do fundacji Mozilla odpowiedniego kandydata do migracji, który spełniałby wcześniej zdefiniowane kryteria. W procesie tym wykorzystano dwa narzędzia. Pierwszym z nich był Searchfox, narzędzie które indeksuje kod źródłowy oraz umożliwia szybkie wyszukiwanie kodu i plików źródłowych. Drugim narzędziem była Bugzilla, system śledzenia zadań Mozilli, służący do zarządzania zgłoszeniami błędów, propozycjami zmian i zadaniami deweloperskimi.
Przy użyciu tych narzędzi oraz na podstawie powyżej opisanych kryteriów do migracji została wybrana wewnętrzna biblioteka testcrasher.dll. Na decyzję dodatkowo wpłynął fakt istnienia w systemie śledzenia błędów Mozilli zadania o numerze Bug 1798688\cite{bug1798688}, które jawnie definiuje cel jako "Replace breakpad with rust-minidump in the testcrasher library" \footnote{Zastąpienie breakpad przez rust-minidump w blibliotece testcrasher}.
\subsection{Cel biblioteki testcrasher}
Biblioteka dynamiczna testcrasher jest narzędziem deweloperskim które używane jest jako tester działania komponentu Crashreporter - wewnętrznego mechanizmu przeglądarki Firefox, odpowiedzialnego za zbieranie i raportowanie informacji o awariach aplikacji.
Działa poprzez wywoływanie awarii procesu a następnie analizę pliku zrzutu pamięci (.dmp) który został wytworzony przez moduł Crashreporter.
Jej działanie skupia się na dwóch obszarach:
\begin{itemize}
    \item \textbf{Analiza zrzutu pamięci:} Za tą część odpowiada plik dumputils.cpp. API tej części zawiera:
    \begin{itemize}
        \item \textbf{DumpHasStream()} - Zwraca wartość \texttt{true}, jeśli dany zrzut pamięci zawiera strumień określonego typu. 
        \item \textbf{DumpHasInstructionPointerMemory()} - Zwraca wartość \texttt{true}, jeśli dany zrzut pamięci zawiera region pamięci który zawiera wskaźnik instrukcji z rekordu wyjąktu.
        \item \textbf{DumpCheckMemory()} - Sprawdza, czy zrzut pamięci zawiera region rozpoczynający się pod adresem określonym w pliku \texttt{crash-addr} w bieżącym katalogu roboczym. Region ten musi mieć długość 32 bajtów i zawierać wartości od 0 do 31 w porządku rosnącym.
    \end{itemize}
    \item \textbf{Wywoływanie awarii procesu:} Za tą część odpowiada plik nsTestCrasher.cpp. API tej części zawiera:
    \begin{itemize}
        \item \textbf{Crash()}
        \item \textbf{EnablePHC()}
        \item \textbf{GetWin64CFITestFnAddrOffset()}
        \item \textbf{TryOverrideExceptionHandler()}
        \item \textbf{SaveAppMemory()}
    \end{itemize}
\end{itemize}
\subsection{Architektura biblioteki testcrasher przed migracją}
Rdzeniem biblioteki dynamicznej testcrasher są dwa pliki źródłowe C++: dumputils.cpp i nsTestCrasher.cpp.
Moduł dumputils.cpp  wykorzystuje do swojego działania zewnętrzną bibliotekę Google Breakpad. Jest to projekt open-source napisany w C++, dostarczający API do obsługi plików minidump. Jednym z głównych zadań tej pracy jest pozbycie się tej zależności.
\\
Kod źródłowy biblioteki korzysta w niewielkim stopniu z kluczowych mechanizmów i konwencji specyficznych dla bazy kodu Firefoksa. Powoduje to że migracja kodu z Języka C++ na Język Rust nie będzie wiązać się z przepisywaniem/modyfikacjami kodu poza biblioteką testcrasher.
\\
System budowania Mozilli, w przypadku biblioteki testcrasher stosuje metodę gdzie pliki .cpp są łączone w jedną jednostkę kompilacji w celu przyspieszenia procesu. Następnie skompilowany kod obiektowy jest linkowany w ostateczną bibliotekę dynamiczną testcrasher.dll.
\\
Głównym sposobem użycia biblioteki sa zautomatyzowane testy, najczęściej pisane w JavaScripcie i uruchamiane w specjalnym frameworku testowym Mozilli(xpcshell-test). Skrypt testowy wywołuje wyeksportowaną funkcję z testcrasher.dll w celu spowodowania awarii i sprawdzenia czy moduł Crash Reporter wygenerował poprawny plik zrzutu pamięci.
\begin{figure}[H]
\centering
\includegraphics[width=0.8\linewidth]{figures/testcrasher.png}
\caption{Architektura biblioteki testcrasher - opracowanie własne}
\label{fig:testcrasher-architecture}
\end{figure}

\section{Strategia migracji i wykorzystane narzędzia}
Kolejnym ważnym aspektem był dobór odpowiedniej strategii oraz narzędzi. W związku z tym kierowaliśmy się ?. 
\subsection{Strategia migracji biblioteki testcrasher}
Przyjęta strategia migracji opiera się na stopniowym i iteracyjnym zastępowaniu kodu C++ kodem Rust, przy jednoczesnym zachowaniu w pełni kompatybilnego publicznego API.
Fundamentalnym założeniem jest, że z perspektywy klienta biblioteki, którym w tym przypadku są skrypty testujące, proces migracji jest całkowicie transparentny. Wymaga to utrzymania stabilnego interfejsu binarnego aplikacji (ABI) zgodnego z językiem C. Dzięki temu poszczególne funkcje, a docelowo całe moduły zaimplementowane w C++, mogą być zastępowane ich odpowiednikami w Rust, a następnie weryfikowane za pomocą istniejącego zestawu testów.
Proces migracji został zaplanowany w następujących, logicznie następujących po sobie etapach:
\begin{enumerate}
    \item \textbf{Konfiguracja procesu budowania:} Pierwszym krokiem jest modyfikacja systemu budowania Mozilli (moz.build) w celu umożliwienia współistnienia kodu C++ i Rust. Polega to na zdefiniowaniu reguł kompilacji dla nowego kodu Rust do postaci biblioteki statycznej (rust\_testcrasher.lib). Następnie, ta biblioteka statyczna jest dołączana do finalnej biblioteki dynamicznej (testcrasher.dll), a jej publiczne symbole są eksportowane w taki sposób, aby zachować zgodność z oryginalnym API.
    \item \textbf{Iteracyjna migracja logiki analitycznej (dumputils.cpp):} Proces właściwej migracji rozpoczyna się od komponentów analitycznych. Poszczególne funkcje odpowiedzialne za parsowanie i analizę plików minidump są reimplementowane w języku Rust, wykorzystując do tego celu crate rust-minidump. Po zaimplementowaniu każdej funkcji w Rust, jej oryginalna wersja w C++ jest usuwana, a nowa implementacja zostaje zintegrowana w procesie budowania.
    \item \textbf{Migracja logiki inicjującej awarie (nsTestCrasher.cpp):} Po pomyślnej weryfikacji poprawności działania modułu analitycznego, analogiczny proces jest stosowany do kodu odpowiedzialnego za inicjowanie stanów awaryjnych. Funkcje C++ są zastępowane przez ich odpowiedniki w Rust, które wykorzystują bibliotekę sadness-generator.
    \item \textbf{Finalizacja i czyszczenie konfiguracji:} Po zakończeniu migracji całości kodu funkcjonalnego do języka Rust, oryginalne pliki źródłowe C++ (dumputils.cpp oraz nsTestCrasher.cpp) są ostatecznie usuwane z drzewa projektu. Konfiguracja w pliku moz.build jest upraszczana, eliminując reguły dotyczące kompilacji C++ dla tej biblioteki.
\end{enumerate}
Dzięki takiemu podejściu, w dowolnym momencie procesu migracji biblioteka pozostaje w pełni funkcjonalna, zawierając mieszankę działającego kodu C++ i Rust.

\subsection{Wykorzytane narzędzia i technologie}
Do przeprowadzenia migracji wybranego fragmentu kodu użyto:
\begin{itemize}
    \item \textbf{Biblioteki Rust(Crate):}
    \begin{itemize}
        \item \textbf{rust-minidump:} Kolekcja bibliotek służąca do odczytu i analizy plików zrzutu pamięci, modelowana na podstawie Google Breakpad\cite{rust-minidump}. W celu przeprowadzenia migracji użyta została biblioteka minidump-processor.
        \item \textbf{sadness-generator:} Biblioteka służąca do intencjonalnego wywoływania awarii procesu na różne sposoby, przykładowo stackoverflow.
    \end{itemize}
\end{itemize}
\subsection{System budowania Mozilla Firefox}
\section{Migracja części odpowiedzialnej za anlizę zrzutów pamięci}
\section{Migracja fragmentu odpowiedzialnego za wywoływanie awarii procesu}
\section{Architektura biblioteki testcrasher po migracji}

