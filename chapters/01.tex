\chapter{Wprowadzenie}
Na początku był c++\\
rust\\
Problem legacy code - duzej bazy kodu nie mozna zmigrowac..
\section{Cel pracy}
Celem pracy dyplomowej jest analiza oraz praktyczna realizacja migracji fragmentu programu udostępnionego przez fundację Mozilla z języka C++ do języka Rust w celu oceny korzyści związanych z bezpieczeństwem i nowoczesnością kodu.
\section{Przegląd rozdziałów}
W rozdziale drugim pracy przedstawiono istniejące rozwiązania dotyczące migracji kodu z języka C++ do języka Rust, ze szczególnym uwzględnieniem projektów realizowanych przez społeczność open source oraz inicjatyw fundacji Mozilla, które ilustrują praktyczne podejścia i narzędzia wspierające ten proces.
\\
W rozdziale trzecim szczegółowo przedstawiono praktyczny proces migracji. Opis rozpoczyna się od omówienia kryteriów, które doprowadziły do wyboru migrowanego fragmentu kodu, a następnie prezentuje jego dogłębną analizę, przyjętą strategię działania, użyte narzędzia oraz finalny, wieloetapowy proces zastępowania kodu C++ kodem Rust.
\\
W rozdziale czwartym podjęto się analizy i testów uzyskanych rezultatów oraz przedstawiono techniczne aspekty pracy. Rozpoczęto od omówienia środowiska testowego i opisano przyjętą metodologię. W kolejnych sekcjach przeprowadzono testy funkcjonalne, których celem było potwierdzenie poprawnego działania kodu. Porównano ze sobą również implementacje w C++ i Rust, biorąc pod uwagę między innymi rozmiar pliku wynikowego oraz czas kompilacji. Rozdział zakończono opisem problemów napotkanych w trakcie przeprowadzania migracji.
\\
W rozdziale piątym przedstawiono podsumowanie wykonanych prac oraz sformułowano wnioski końcowe dotyczące korzyści płynących z zastosowania języka Rust. Wskazano również potencjalne kierunki dalszego rozwoju projektu.

