\chapter{Testowanie i wnioski końcowe}
Testy zostały przeprowadzone w następującym środowisku deweloperskim, przedstawionym poniżej:
\begin{itemize}
    \item {System operacyjny:} Windows 10 Pro 22H2 64-bit
    \item {Procesor:} AMD Ryzen 5 7600X (6 × 4.70 GHz)
    \item {Pamięć RAM:} 32 GB
    \item {rustc:} 1.91.0
    \item {cargo:} 1.91.0
    \item {clang:} 19.1.5
    \item {Firefox:} 144.0.1 (release)
\end{itemize}
W ramach testów funkcjonalnych oceniano poprawność działania biblioteki testcrasher po migracji z języka C++ do Rust. Uruchomiono dwa zestawy testów:
\begin{itemize}
    \item \textbf{Testy jednostkowe} \texttt{./mach test toolkit/crashreporter/test/unit/}
    \item \textbf{Testy IPC} \texttt{./mach test toolkit/crashreporter/test/unit\_ipc/}
\end{itemize}
Testom poddano dwa główne obszary funkcjonalne biblioteki:
analiza zrzutu pamięci: \textbf{DumpHasStream()}, \textbf{DumpHasInstructionPointerMemory()}, \textbf{DumpCheckMemory()} oraz
wywoływanie awarii procesu: \textbf{Crash()}, \textbf{EnablePHC()}, \textbf{GetWin64CFITestFnAddrOffset()}, \textbf{TryOverrideExceptionHandler()}, \textbf{SaveAppMemory()}.
Dodatkowo weryfikowano brak regresji, rozumiany jako zachowanie niezmienionego ABI biblioteki, identyczne zachowanie funkcji eksportowanych oraz przejście wszystkich testów XPCShell bez modyfikacji po stronie testów.
W ramach analizy porównano również czas kompilacji oraz rozmiar wygenerowanej biblioteki przed i po migracji. \\
Do obliczenia czasu kompilacji oraz rozmiaru pliku biblioteki testcrasher zostały użyte następujące metody pomiarowe: czas kompilacji, raportowany przez \texttt{mach}, oraz rozmiar pliku, raportowany przez PowerShell: \texttt{(Get-Item "testcrasher.dll").Length}.
\section{Weryfikacja funkcjonalna}
Wyniki testów przed i po migracji były identyczne, co potwierdza, że implementacja w języku Rust zachowała pełną kompatybilność funkcjonalną z wersją C++.
\begin{itemize}
    \item {Testy jednostkowe:}
    \begin{lstlisting}{language = bash}
    Ran 48 checks (48 tests)
    Expected results: 43
    Skipped: 4 tests
    Unexpected results: 1 (1 fail)
    \end{lstlisting}
    \item {Testy IPC:}
    \begin{lstlisting}{language = bash}
    Ran 10 checks (10 tests)
    Expected results: 10
    Unexpected results: 0
    OK
    \end{lstlisting}
\end{itemize}
Takie same wyniki uzyskano po migracji biblioteki na Rust. Oznacza to, że wszystkie ścieżki funkcjonalne zostały zachowane oraz ABI biblioteki pozostało zgodne.
\section{Analiza czasu kompilacji i rozmiaru pliku}
\subsection{Czas kompilacji}
Poniżej przedstawiono wyniki testów porównawczych dotyczących czasu kompilacji biblioteki testcrasher przed i po migracji:
\begin{table}[!ht]
    \centering
    \small
    \caption{Czas kompilacji biblioteki \texttt{testcrasher} przed i po migracji}
    \label{tab:compile-time}
    \begin{tblr}{%
        hlines,
        vlines,
        row{1}={font=\bfseries},
        column{1}={halign=c},
        column{2}={halign=c},
        column{3}={halign=c},
    }
        Wersja & Czas & Zmiana \\
        Przed migracją & 0:00.90 & -- \\
        Po migracji   & 0:01.23 & +36\% \\
    \end{tblr}
\end{table}
\\
Wzrost czasu wynika bezpośrednio z charakterystyki kompilatora Rust: analiza typów jest znacząco bardziej kosztowna obliczeniowo, Rust stosuje intensywną optymalizację kodu, generowanie jest często wolniejsze niż w przypadku C++.\cite{rustc-overview}
Choć czas kompilacji wzrósł o 36\%, jest to niewielka różnica w skali całej przeglądarki Firefox. W praktyce zdecydowana różnica czasowa staje się pomijalna przy ciągłym rozwoju.
\subsection{Rozmiar pliku biblioteki}
Poniżej przedstawiono wyniki testów porównawczych dotyczących rozmiaru pliku biblioteki testcrasher przed i po migracji:
\begin{table}[!ht]
    \centering
    \small
    \caption{Rozmiar biblioteki \texttt{testcrasher} przed i po migracji}
    \label{tab:binary-size}
    \begin{tblr}{%
        hlines,
        vlines,
        row{1}={font=\bfseries},
        column{1}={halign=c},
        column{2}={halign=c},
        column{3}={halign=c},
    }
        Wersja & Rozmiar & Zmiana \\
        Przed migracją  & 167\,936 B & – \\
        Po migracji   & 455\,680 B & +271\% \\
    \end{tblr}
\end{table}
\\
Znaczne zwiększenie rozmiaru biblioteki wynika z charakterystyki kompilatora Rust, który statycznie dołącza część niezbędnych elementów bibliotecznych, generuje bardziej rozbudowane konstrukcje wewnętrzne, wynikające z modelu bezpieczeństwa pamięci oraz typów opisujących błędy, nie korzysta z udostępnianych przez system współdzielonych bibliotek, co skutkuje większą ilością kodu dołączanego do finalnego pliku w porównaniu do kompilacji C++.\cite{rustlib}\cite{linking-rust-crates}
W praktyce zwiększenie rozmiaru z 167 KB do 455 KB nie ma wpływu na działanie systemu.

\section{Napotkane problemy i ograniczenia}
W trakcie migracji zidentyfikowano następujące problemy i ograniczenia:
\begin{itemize}
    \item \textbf{Różnice w obsłudze błędów.} W C++ błędy sygnalizowane są przez nullptr i wartości logiczne, podczas gdy w Rust wymuszona jest obsługa Result<T, E> oraz Option<T>.
    \item \textbf{Różnice między Breakpad a rust-minidump.} API rust-minidump nie jest bezpośrednim odpowiednikiem Google Breakpad, co wymagało zmiany sposobu parsowania zrzutów pamięci.
    \item \textbf{Długi czas kompilacji.} Pełna kompilacja projektu na podanym systemie trwa ok. 30 minut. Możliwe jest kompilowanie wyłącznie modułów, jednak nie zawsze jest to wystarczające, szczególnie w przypadku zmian ABI lub plików moz.build.
\end{itemize}
\section{Wnioski końcowe}
Przeprowadzona migracja fragmentu biblioteki testcrasher z języka C++ do Rust pozwoliła na praktyczną weryfikację przedstawionych założeń pracy. Zgodnie z celem pracy, polegającym na ocenie możliwości zastąpienia kodu typu legacy nowoczesnym i bezpieczniejszym kodem Rust, uzyskane wyniki potwierdzają, że taka migracja jest jak najbardziej możliwa oraz uzasadniona.
Podsumowując, migracja analizowanego modułu potwierdza, że Rust jest realną i skuteczną alternatywą dla C++, może być wdrażany stopniowo w dużych projektach, zapewnia korzyści w zakresie bezpieczeństwa pamięci i jakości kodu oraz nie zakłóca stabilności ani kompatybilności całego systemu.
Otrzymane wyniki potwierdzają zasadność stosowania Rust w projektach o wysokim znaczeniu dla bezpieczeństwa, takich jak Mozilla Firefox, oraz wskazują kierunek dla dalszej modernizacji istniejących komponentów.