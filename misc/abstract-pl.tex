\chapter*{Streszczenie}

W niniejszej pracy znajduje się opis procesu migracji kodu należącego do fundacji Mozilla z języka C++ na język Rust.
Przedmiotem praktycznej realizacji jest biblioteka dynamiczna testcrasher, wykorzystywana w procesie testowania mechanizmów raportowania błędów w przeglądarce Firefox. Głównym celem pracy jest wykonanie takiej migracji oraz ocena korzyści związanych z bezpieczeństwem pamięci i nowoczesnością kodu, jakie oferuje język Rust.
Oznacza to porównanie kodu przed i po migracji pod kątem zarządzania pamięcią, szybkością działania oraz utrzymania oprogramowania.
Przedstawione zostają kolejne kroki podjęte w celu przeprowadzenia migracji. Najpierw wyznaczono kryteria według których dokonano wyboru fragmentu kodu, kolejne sekcje obejmują jego szczegółowa analiza i jego stopniowe przepisanie na język Rust. 
Wyniki prac potwierdziły, że możliwa jest stopniowa wymiana kodu C++ na Rust w dużym projekcie przy zachowaniu pełnej kompatybilności funkcjonalnej. Nowa implementacja eliminuje ryzyko typowych błędów zarządzania pamięcią dzięki wykorzystaniu modelu własności języka Rust, co zweryfikowano za pomocą testów regresji.

% Define 5-10 keywords which can describe what you're writing about
\bigskip
\noindent
\textbf{Słowa kluczowe:} C++, Rust, migracja kodu, bezpieczeństwo pamięci, Mozilla Firefox

% Choose the field from the OECD list:
% https://en.wikipedia.org/wiki/Fields_of_Science_and_Technology
% With some research, you should find the whole list officially
% traslated into Polish.
% The below should be fine for PG ETI (informatyka)
\bigskip
\noindent
\textbf{Dziedzina nauki i techniki, zgodnie z wymogami OECD:}
elektrotechnika, elektronika, \\ inżynieria informatyczna
